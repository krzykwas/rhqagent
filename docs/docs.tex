\documentclass[a4paper,twoside,11pt]{article}
\usepackage[utf8]{inputenc}
\usepackage{indentfirst,graphicx,amsmath}
\usepackage[left=2cm,top=2cm,right=2cm,nohead]{geometry}
\usepackage{hyperref}
\usepackage{tabularx}
\pagestyle{plain}

\makeatletter
\renewcommand\@seccntformat[1]{\csname the#1\endcsname.\quad}
\renewcommand\numberline[1]{#1.\hskip0.7em}
\makeatother

\author{Krzysztof Kwaśniewski}
\title{\LARGE Documentation for PyAgent}

\begin{document}
	\maketitle

	\tableofcontents
	\newpage

	\section{Summary}
		PyAgent is an agent for RHQ, an enterprise management system. It provides RHQ\footnote{\href{http://rhq-project.org/display/RHQ/Home}{http://rhq-project.org/display/RHQ/Home}} with data gathered from desktop stations scattered across the network so that the stations could be monitored and managed. The data RHQ can handle can be of various types, one refers to them as \emph{metrics}. These may include for instance repetitive measurements of a CPU's temperature over the time.

		To collect this data, PyAgent communicates with CIM servers, which are WBEM\footnote{\href{http://dmtf.org/standards/wbem}{http://dmtf.org/standards/wbem}}-based services. The WBEM standard describes an abstract interface that allows to interact with different computer systems and - for instance - access a data on a aforementioned CPU's temperature (and many more).

		PyAgent is extensible and it's possible to quite easily add to it support for more protocols - so that it could gather the data from different sources, not only WBEM-based, and send it to a different destination, not only RHQ. Although now there does not seem to be an urgent need for that, one day it may turn out to be necessary.

		This project has been started in April/May 2012 as part of The Google Summer Of Code program, mentored by Heiko Rupp\footnote{\href{https://community.jboss.org/people/pilhuhn}{https://community.jboss.org/people/pilhuhn}}.
		
		The code is implemented in Python 2.x; will not work with Python 3.x, since it depends on a python-wbem library to communicate with WBEM-based services. The following list is - I hope - a full list of PyAgent's dependencies:

		\begin{itemize}
			\item python-wbem\footnote{\href{http://pywbem.sourceforge.net/}{http://pywbem.sourceforge.net/}}
			\item python-nose\footnote{\href{http://nose.readthedocs.org/en/latest/}{http://nose.readthedocs.org/en/latest/}}
			\item python-coverage\footnote{\href{http://nedbatchelder.com/code/coverage/}{http://nedbatchelder.com/code/coverage/}}
			\item python-lxml\footnote{\href{http://lxml.de/}{http://lxml.de/}}
			\item python-distutils\footnote{\href{http://pypi.python.org/pypi/setuptools/0.6c11}{http://pypi.python.org/pypi/setuptools/0.6c11}}
			\item make
			\item rpmbuild (to build an RPM package for Fedora)
			\item pdflatex (to produce this documentation)
		\end{itemize}

		This is a free software licensed under the terms of GNU General Public License version 3 or later as published by the Free Software Foundation.

	\section{Running the code}
		To run the code, you need to invoke the \texttt{rhqagent} startup script located in the top project's directory. As a minimum, you need to provide an argument \texttt{-{}-configuration-file} pointing at an xml configuration file, as PyAgent needs to know where it can gather the data from and where should it send it to.

		PyAgent can help you generate such a configuration file, if you also pass a \texttt{-{}-setup} argument to it.

		For more details, please invoke PyAgent with a \texttt{-{}-help} argument.

		You can also build packages, currently an RPM and in the Python Package Index format, so that the code could be distributed easier. After installing such a package, you will be also able to invoke PyAgent with an init script \texttt{rhqagentd}. To do that (at least in both Ubuntu and Fedora), invoke a command \texttt{service rhqpyagentd} to get a list of supported arguments.

	\section{Directory structure}
		To give you an overall impression on how this project is organised, here you have a brief description of the files it consists of.

		\begin{itemize}
			\item COPYING -- contains the text of GPLv3 license under which (or any later) this project is licensed.
			\item docs -- the documentation of PyAgent lives here in a TEX format. To turn it into PDF, run \emph{make docs} in the main project's directory.
			\item Makefile -- lets one perform a couple of actions. One can use it to run unit tests with or without code coverage measurement; create a PDF file with documentation out of this one in a TEX format; build an RPM or a PyPi package and - finally - clean up all the temporary files created during the previous operations.
			\item packages -- RPM and PyPi packages are created in this directory.
			\item pyagent -- contains the source code of the application.
				\begin{itemize}
					\item agent -- a main module of PyAgent that governs all the operations it performs.
					\item data
						\begin{itemize}
							\item model -- PyAgent's data model.
							\item provider -- the abstract definition for classes that are able to provide data for PyAgent and all the implementations derived from it.
							\item sender -- the abstract definition for classes that are able to pass further data gathered by PyAgent and all the implementations derived from it.
						\end{itemize}
					\item exception -- contains definitions of exceptions thrown by PyAgent's submodules.
					\item \_\_init.py\_\_ -- necessary to turn this directory into a Python module (available for importing).
					\item settings -- a module responsible for generating and parsing the configuration file as well as parsing the arguments passed in the command line.
						\begin{itemize}
							\item sample\_settings.xml -- sample configuration file. If you want to define your own configuration file by hand, here you will find an example of such a file.
							\item settings.xsd -- an xml schema for PyAgent's configuration file. The configuration file must comply with this schema.
						\end{itemize}
					\item test -- unit tests live here.
				\end{itemize}
			\item README -- a brief description of the project.
			\item rhqpyagent -- a startup script for PyAgent.
			\item rhqpyagentd -- an init script that turns PyAgent to behave as a daemon (system service), this is the expected way of running PyAgent after its RPM or PyPi package has been installed.
			\item setup.py -- necessary to build a PyPi package, you don't need to use it directly, use Makefile instead.
		\end{itemize}

	\section{How to...?}
		The syntax of the configuration file is somewhat self-explanatory and in the pyagent.settings submodule there is a sample configuration file, I won't go into deep details here, it's not very complicated. On the other hand a couple of things need to be written down.

		\begin{itemize}
			\item Add a new data provider/sender

				As mentioned before, PyAgent was designed in an extensible way. You can quite easily add to it a new data provider (let's focus on that, data senders work in the same way). You do this by subclassing pyagent.data.provider.AbstractDataProvider.AbstractDataProvider. No, this is not a  mistake. The first AbstractDataProvider refers to the module name and the class named in the same way is stored inside.

				Your implementation has to be called according to the schema \texttt{<<PROTOCOL>>DataProvider} and live in a file \texttt{pyagent.data.provider.implementation.<<PROTOCOL>>DataProvider}, where \texttt{<<PROTOCOL>>} is an uppercase name of the protocol for which you're just adding a new provider. For instance, to support FTP, you would create a class FTPDataProvider in pyagent.data.provider.\-implementation.FTPDataProvider. After that, you can put \emph{ftp} as a protocol name in the configuration file and PyAgent will understand what your intentions are.

			\item Add a callback
		\end{itemize}

\end{document}

